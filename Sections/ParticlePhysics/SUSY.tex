\chapter{Supersymmetric Extensions to the Standard Model}

Supersymmetric (SUSY) theories have been proposed as a potential solution to some of the problems described in the previous chapter. According to SUSY~\cite{SUSY_primer}, there is an additional "supersymmetric" partner to each particle that we currently know of. Each boson has a fermionic superpartner, and each fermion has a bosonic superpartner. This simple extension is able to solve the three problems we identified previously.

First, superpartners provide a natural solution to the hierarchy problem. Since each Standard Model fermion and boson has a Higgs mass loop correction which is almost exactly cancelled out by its superpartner, the observed mass of the Higgs boson is able to stay close to zero. Furthermore, if the lightest neutral SUSY particle is incapable of decaying into a normal-matter state, then we would have a long-lived massive particle which does not interact with electromagnetism. This particle could remain undetected and thus would be a candidate for dark matter. In addition, various configurations of string theory state that if SUSY is imposed as a local symmetry then supergravity theories can be formed, merging the Standard Model and general relativity.

With all of these strong points, it would seem that SUSY is a very promising theory. However, despite physicists' best efforts, attempts to search for SUSY particles have so far proved unfruitful. \textbf{History}

Clearly, if SUSY is a correct theory, it must undergo spontaneous symmetry breaking which changes the masses of the undiscovered superpartners, so that e.g. the \textbf{$Z_0$} is not close in mass to its superpartner. Thus it may be that the undiscovered particles are currently outside the energy range of our colliders (though there are upper limits to particle masses before the theory becomes unnatural), or it may be that the particles are too close together in mass, in which case their decay products will be hard to detect. This second possibility will be examined later, and will form part of the motivation for developing our machine-learning based particle identification systems.

\begin{figure}[htbp]
    \centering
    \includegraphics{Images/SUSY.png}
    \caption{Supersymmetric partners of standard-model particles. This diagram displays the squarks, sleptons, and gauginos which make up SUSY. The gauginos (from top to bottom, and left to right) are the gluino, photino, zino, the winos, and the Higgsinos. The photino, zino, and two neutral Higgsinos combine to form mass eigenstates $\tilde{\chi}^0_1, \tilde{\chi}^0_2, \tilde{\chi}^0_3, \tilde{\chi}^0_4$, which are called the neutralinos. The winos and two charged Higgsinos combine to form $\tilde{\chi}^\pm_1, \tilde{\chi}^\pm_2$, the charginos.}
    \label{fig:my_label}
\end{figure}

\textbf{MSSM}