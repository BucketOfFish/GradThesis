\chapter{The History of Understanding the History of Everything}

In the beginning people knew nothing. What is the universe made of? Can you turn a rock into gold? What happens if you get smaller and smaller - is there a whole different world down there? These were just a small subset of questions our early ancestors would never know the answer to, since they all died out before the development of particle physics in the $20^{th}$ century. But let's entertain ourselves by examining a few of the early mental models they had, and how the persistent desire to improve these models led to the development of modern physics.

It's somewhat strange at first glance that many ancient cultures had roughly the same classical elements, but upon examination of the things commonly found in nature, we can understand where these groupings came from. Most early building materials, food, and organic compounds came out of and into the earth. These things seemed to cycle endlessly and turn easily into each other, so the element of earth or clay logically served as a catch-all building block for all such materials. Chinese philosophy further divided wood and metal into their own categories separate from earth. Next came fire and water, distinct enough from earth to warrant separate categories of their own. Together, earth, water, and fire also roughly corresponded to the common states of matter (solid, liquid, gas/plasma), so it's understandable that these mental clusterings became common in many early philosophies. The elements of air and aether were somewhat unique to Greek thought however, and were probably later transported into India and folded into the classic Buddhist elements.

In all then, we had air, water, earth, fire, and aether in Greek philosophy, corresponding to the five Platonic solids~\cite{Plato}, and according to Empedocles~\cite{Empedocles}, bound together by love and strife. In Indian philosophy we had air, water, and earth, corresponding to different categories of food~\cite{Chandogya}. This evolved with Greek influence into the five elements of early Buddhism and Hinduism, each with their own associations~\cite{friesian}. In Chinese philosophy we had metal, wood, earth, water, and fire~\cite{Tao}. When combined with air and aether, the system was further expanded to form associations with the sun, moon, and five visible planets~\cite{friesian}.

And so we see how throughout the ancient world, a complex series of inter-related, "common-sense" memetic systems came to dominate early philosophy. Based on pattern recognition and concept association rather than experimental evidence, these models guided intellectual pursuits for many centuries. There is a tendency among physicists today to trace the beginnings of atomic theory back to the Greek philosopher Democritus, but without the use of scientific experimentation guided by mathematical modelling, his philosophy had no more value than these other ancient systems, aside from the somewhat irrelevant distinction of ultimately being correct.

It was not until the chemical revolution of the $16^{th}$ to $18^{th}$ centuries that we had (from a modern standpoint) acceptable proof of the existence of tiny building blocks of nature. Careful experiments carried out during this time allowed scientists to measure the masses of compounds before and after chemical reactions. Various elements were separated for the first time, and the role of oxygen in combustion was discovered. In his book "New System of Chemical Philosophy", John Dalton compiled relative masses of known elements, providing evidence for the existence of atoms~\cite{new_system_chemical}. Dalton was also able to use his system to calculate the composition of elemental gasses in the atmosphere. Einstein's later work on Brownian motion in 1905 provided further evidence that atoms and molecules exist.

Feynman argued in his Lectures on Physics~\cite{feynman1965flp} that the existence of atoms was the most important discovery in modern science, from which most other knowledge could be derived. However, demonstrating the existence of atoms was not the end of the line for tiny physics. The Rutherford gold foil experiment~\cite{Rutherford}, first performed in 1908, demonstrated the existence of substructure in the atom. By shooting alpha particles at gold foil and by measuring the rate and angles of scattering, these researchers were able to demonstrate that most of the mass of an atom was contained in a concentrated volume in the nucleus. Further work by many scientists led to the separation of the proton, neutron, and electron by the 1930's.

Over the next few decades, the discovery of multiple types of new hadrons in studies of cosmic radiation led to the proliferation of a so-called "particle zoo". The quark model proposed by Gell-Mann and Zweig put forward a theory of subatomic particles which could describe the observed phenomena, and the existence of such subatomic structure was observed via deep inelastic scattering experiments performed by SLAC in 1968~\cite{SLAC_quark}.

Thus from the beginning of science we have probed the structure of the universe by breaking things apart and examining the constituents. First the components of air and wood and candle wax were whirled and separated and taken apart. Then the atoms were collided, discovering the contents of the nucleus. Then the protons, neutrons, and electrons were smashed together at high energies, finding further substructure in the proton and neutron (though the electron is still elementary as far as we know). Even today we find new knowledge by smashing these components together at higher and higher energies. By examining the results of trillions of collisions, we make more precise probes of what particles exist at the smallest length scales, and how they interact with each other. Through the process of all this smashing, we learned details about the functioning of the four fundamental forces of the universe - the electromagnetic, weak, strong, and gravitational forces.

All this probing has allowed us to test theories and make measurements of particle interactions to extreme limits of precision. The result of all this work is the Standard Model of Physics, which will be described in the next section.