\chapter{The Standard Model of Physics}\label{chap:SM}

\section{The Theory As It Stands}

\subsection{Putting It All Together}

The purpose of the LHC and other similar colliders is to study physics at the most fundamental level of existence. We are interested in probing and measuring the smallest building blocks of nature, and in discovering how they go together. The most complete picture which we have to date is shown on the left side of Figure~\ref{SUSY}. In this model, known as the standard model, we describe the universe as a composition of leptons, quarks, and force carriers. Leptons include the well-known electron, as well as its nearly-invisible partner, the electron neutrino. Quarks include the building blocks of protons and neutrons, the up and down quarks. Furthermore, there are the heavier generations of these particles, which are just like the electron, electron neutrino, up quark, and down quark, only with more mass. We also have the force carriers (or gauge bosons) - photons for the electromagnetic force, gluons for the strong force, and W and Z bosons for the weak force. Finally, completing the picture is the recently-discovered Higgs boson, an excitation of the Higgs field which gives mass to the other fundamental particles. There are also the antimatter counterparts to these particles, which have opposite charge, though the neutral force carriers are their own antiparticles.

%\begin{figure}[t]
%    \centering
%    \includegraphics[width=0.7\linewidth]{images/standard_model.png}
%    \caption{The standard model.}
%    \label{standard_model}
%\end{figure}

\begin{figure}[t]
    \centering
    \includegraphics[width=0.7\linewidth]{images/SUSY.png}
    \caption{(left) The standard model. (right) Supersymmetric partners of standard-model particles. This diagram displays the squarks, sleptons, and gauginos which make up SUSY. The gauginos (from top to bottom, and left to right) are the gluino, photino, zino, the winos, and the Higgsinos. The photino, zino, and two neutral Higgsinos combine to form mass eigenstates $\tilde{\chi}^0_1, \tilde{\chi}^0_2, \tilde{\chi}^0_3, \tilde{\chi}^0_4$, which are called the neutralinos. The winos and two charged Higgsinos combine to form $\tilde{\chi}^\pm_1, \tilde{\chi}^\pm_2$, the charginos.}
    \label{SUSY}
\end{figure}

This model has produced some of the most accurate agreements between theory and experiment ever achieved in science, but we know that it does not describe the entire universe. For instance, the standard model only describes ordinary matter and antimatter, but these only account for about $5\%$ of the mass in the universe. Another $27\%$ is dark matter, and the remaining $68\%$ is dark energy. Gravity is not included in the standard model, as it remains the least-understood fundamental force. Many questions about it remain to be answered, such as why gravity is so much weaker than all the other forces. There is also the Higgs hierarchy problem, which relates to why the Higgs boson has the mass that it does \cite{hierarchy}. If we calculate the expected mass of the Higgs, we find that quantum loop corrections from virtual particles ought to push the mass up to an order of magnitude around the breakdown point of known physics, the Planck mass at $m_{Planck} = 10^{18}$ GeV. One way the Higgs can have a mass around 126 GeV is for bosonic and fermionic loop corrections to the Higgs mass (which have opposite sign) to cancel each other out.

\subsection{The Final Piece of the Puzzle}

~\cite{Griffiths}

\section{Physics Is Not Over}

\subsection{Problems with the Standard Model}

\subsection{What Particle Physicists Do All Day}