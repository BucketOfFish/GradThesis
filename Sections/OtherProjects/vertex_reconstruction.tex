\chapter{Vertex Reconstruction}

In the next decade, the planned High Luminosity Large Hadron Collider (HL-LHC) upgrade~\cite{Apollinari:2284929} will enhance
the experimental sensitivity to rare phenomena by increasing the number of collected proton-proton collisions by a factor of ten.

First, I worked on investigating methods for improving vertex reconstruction for use in future high-pileup environments, as part of the vertex reconstruction group at ATLAS. The approach we used was based on dividing physical space into a grid of three-dimensional pixels, and using reconstructed tracks to "fill in" those pixels and generate a 3D image. The resulting image was then Fourier transformed, filtered to remove high frequencies, reverse Fourier transformed, and collapsed along the beam axis, resulting in a 1D function where the peaks corresponded to potential vertex locations (vertex "seeds"). Tracks were then associated with their nearest seeds, and from each cluster of tracks we fit a final vertex. I performed performance comparisons between the new and old vertexing methods, and did studies based on tuning parameters of the new method. Some of these results were shown in my presentation at the 2015 ATLAS CP Tracking Workshop \cite{vertex}. I also performed studies on using machine-learning methods to better associate tracks with vertex seeds, and to recognize and recombine "split" vertices, though these methods did not show significant improvements over baseline during my time with the project.