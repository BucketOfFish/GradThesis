\chapter{FTk}

Furthermore, I have contributed some work to the fast tracker (FTk) upgrade for ATLAS \cite{FTk}. The FTk is an FPGA-based method for performing online track reconstruction, for use in the inner detector. I worked on the extrapolator board with Professor Mark Neubaeur's team. The extrapolator is responsible for reconstructing 12-layer tracks (for the 12 layers of the inner detector) using preliminary 8-layer tracks and a collection of hits from the other 4 layers. I developed a new memory storage mechanism for organizing and retrieving hits, and have recently completed a rewrite of the extrapolation code. We are currently working to ensure uninterrupted data flow, and are attempting to optimize our firmware to run at 200 MHz.

%%%%%%%%%%%%%%%

The LHC currently produces about 40 collisions per beam-crossing at 13 TeV center-of-mass energy, but with the proposed high-luminosity LHC (HL-LHC) upgrade there are plans to increase these numbers to 140 collisions per beam-crossing at 14 TeV by 2026, amounting to a total of 250 $fb^{-1}$ of data per year, with ultimate plans to increase to 200 collisions per beam-crossing. During the HL-LHC upgrade, ATLAS similarly plans to upgrade its detection capabilities via the ATLAS Phase-II upgrade, focusing on items such as replacing the inner detector with an all-silicon-based inner tracker (ITk), improving data acquisition, and increasing the granularity of calorimeters \cite{ATLAS_phaseII}. The higher data rates and improved calorimeters provide an incentive to develop and implement the machine-vision based particle identification systems described later in this paper. In the interest of improving upcoming SUSY searches, I have aided in the development of several components in the ATLAS detector. Though they will not be the focus of this paper, as they mostly impact future analyses, I would like to mention them in passing.

First, I worked on investigating methods for improving vertex reconstruction for use in future high-pileup environments, as part of the vertex reconstruction group at ATLAS. The approach we used was based on dividing physical space into a grid of three-dimensional pixels, and using reconstructed tracks to "fill in" those pixels and generate a 3D image. The resulting image was then Fourier transformed, filtered to remove high frequencies, reverse Fourier transformed, and collapsed along the beam axis, resulting in a 1D function where the peaks corresponded to potential vertex locations (vertex "seeds"). Tracks were then associated with their nearest seeds, and from each cluster of tracks we fit a final vertex. I performed performance comparisons between the new and old vertexing methods, and did studies based on tuning parameters of the new method. Some of these results were shown in my presentation at the 2015 ATLAS CP Tracking Workshop \cite{vertex}. I also performed studies on using machine-learning methods to better associate tracks with vertex seeds, and to recognize and recombine "split" vertices, though these methods did not show significant improvements over baseline during my time with the project.

I also performed some work on upgrades to the ITk for the ATLAS Phase-II upgrade. Recently I spent a year at Argonne under the guidance of Dr. Jessica Metcalfe working on manufacturing methods for mass production of silicon pixel detectors in preparation for the upgrade. During that time I worked on setting up a manufacturing lab and various electronic test environments, on assembling pixel modules using various mechanical epoxy-based techniques, on designing adapter boards for readout electronics, and on performing tests on assembled modules. I also participated in a three-month testbeam at Fermilab with the University of Geneva, during which we gathered and analyzed data from a proposed HVCMOS pixel detector. The idea behind this detector was that due to advances in silicon sensor manufacturing, we could build readout circuitry directly into the detector, removing the need for an external readout chip. This resulted in less detector material, and also reduced manufacturing cost. I presented results from this testbeam at DPF 2017 \cite{DPF}, and this analysis, combined with additional data taken by the University of Geneva at the SPS in CERN, will be presented in an upcoming paper on the new detector.

Furthermore, I have contributed some work to the fast tracker (FTk) upgrade for ATLAS \cite{FTk}. The FTk is an FPGA-based method for performing online track reconstruction, for use in the inner detector. I worked on the extrapolator board with Professor Mark Neubaeur's team. The extrapolator is responsible for reconstructing 12-layer tracks (for the 12 layers of the inner detector) using preliminary 8-layer tracks and a collection of hits from the other 4 layers. I developed a new memory storage mechanism for organizing and retrieving hits, and have recently completed a rewrite of the extrapolation code. We are currently working to ensure uninterrupted data flow, and are attempting to optimize our firmware to run at 200 MHz.