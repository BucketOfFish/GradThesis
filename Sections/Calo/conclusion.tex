\chapter{Conclusion and Future Work}\label{sec:conclusion}

This study showed how deep learning techniques could outperform traditional and resource-consuming techniques in  tasks typical of physics experiments at particle colliders, such as particle shower simulation and reconstruction in a calorimeter. We considered several model architectures, notably 3D convolutional neural networks, and showed competitive performance, matched to short execution time. In addition, this strategy comes with a GPU-friendly computing solution and would fit the current trends in particle physics towards heterogeneous computing platforms.

We confirmed findings from previous studies of this kind. On the other hand, we did so utilizing a fully accurate detector simulation, based on a complete GEANT4 simulation of a full particle detector, including several detector components, magnetic field, etc. In addition, we designed the network so that different tasks are performed by a single architecture, optimized through an hyperparameter scan.

We look forward to the development of similar solutions for current and future particle detectors, for which this kind of end-to-end solution could be extremely helpful.