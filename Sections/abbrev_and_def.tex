\chapter*{List of Abbreviations and Definitions}

\begin{symbollist*}
\item[$\eta$] Also known as pseudorapidity. $\eta=-ln(tan(\theta/2))$. $\eta$ is typically used rather than $\theta$ because differences in pseudorapidity are relativistically invariant to boosts along the beam axis.
\item[$\theta$] Angle measured from the beam axis.
\item[ATLAS] A Toroidal LHC ApparatuS. A detector at the LHC.
\item[background] Expected events from known components of a theory. See Chapter~\ref{sec:background}.
\item[baryon] A composite particle made of three quarks. Protons and neutrons are examples of baryons.
\item[beam] A stream consisting of billions of particles. Two beams are circulated in opposite directions in a circular collider and made to cross at a collision point. Beams consist of many separated bunches.
\item[beam axis] Line formed by the path of a beam through the detector. The two colliding beams in a detector are actually at a very small angle to each other when they cross, but the average is taken as the beam axis.
\item[boson] A particle with integer spin. Examples include photons, W and Z bosons, and mesons.
\item[bunch] A tight cluster of particles in a beam. Each bunch in the LHC is typically several centimeters long and a few micrometers wide.
\item[bunch crossing] The collision of two bunches in a detector, resulting in a single event.
\item[CERN] Conseil Européen pour la Recherche Nucléaire. A multinational high-energy physics experiment operating on the French-Swiss border.
\item[data] Events reconstructed from actual collisions, as opposed to MC events.
\item[decay] The spontaneous transformation of one particle into two or more particles.	
\item[event] Detector recording of a single bunch-bunch collision.
\item[excess] When the number of measured events are greater than the number of expected events from background. Excesses are measured in sigmas, or the number of standard deviations above background.
\item[exclusion] A model is excluded if a search finds that the model is less likely to have produced measured data than the Standard Model by some statistical likelihood.
\item[fermion] A particle with half-integer spin (e.g. $1\frac{1}{2}$). These include all leptons and baryons.
\item[FTk] Fast TracKer. A hardware-based track reconstructor which is part of the ATLAS trigger system.
\item[hadron] A composite particle made of two or more quarks. Includes all mesons and baryons.
\item[HEP] High energy physics. Also known as particle physics.
\item[lepton] Electrons, muons, taus, and their associated neutrinos and antiparticles.
\item[LHC] Large Hadron Collider. A circular proton-proton collider operating at CERN.
\item[meson] A composite particle made of two quarks.
\item[MC] Monte Carlo. A class of algorithms based on random sampling. In high-energy physics, MC events (as opposed to data events) are simulated via MC algorithms, rather than gathered from actual collisions.
\item[model] An HEP model describes which particles exist in nature and how they interact. e.g. the Standard Model.
\item[pT] Transverse momentum. The radial component of an object's momentum, as opposed to the components in the beam axis or azimuthal directions.
\item[RNN] Recurrent neural net. A neural architecture described in Section~\ref{sec:RNN}.
\item[search] An attempt to find evidence that collision data is better described by an alternative model than by the Standard Model. e.g. SUSY search.
\item[SM] Standard Model. This model describes how the electromagnetic, weak, and strong forces interact with all known elementary particles, and is described in Chapter~\ref{chap:SM}.
\item[SUSY] Supersymmetry. A class of models described in Chapter~\ref{chap:SUSY}.
\item[trigger] A hardware or software-based set of criteria which must be passed in order for an event to be saved. E.g. the presence of at least one particle in the event with pT $>$ 10 GeV.
\end{symbollist*}

\item[cross section]
\item[ET]
\item[FPGA]
\item[hard scatter]
\item[jet]
\item[MET]
\item[parton]
\item[pileup] \cite{pileup}