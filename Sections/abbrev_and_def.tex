\chapter*{List of Abbreviations and Definitions}

\begin{symbollist*}
\item[ATLAS] A Toroidal LHC ApparatuS. A detector at the LHC.
\item[baryon] A composite particle made of three quarks. Protons and neutrons are examples of baryons.
\item[boson] A particle with integer spin. Examples include photons, W and Z bosons, and mesons.
\item[CERN] Conseil Européen pour la Recherche Nucléaire. A multinational high-energy physics experiment operating on the French-Swiss border.
\item[data] Events reconstructed from actual collisions, as opposed to MC events.
\item[fermion] A particle with half-integer spin (e.g. $1\frac{1}{2}$). These include all leptons and baryons.
\item[FTk] Fast TracKer. A hardware-based track reconstructor which is part of the ATLAS trigger system.
\item[hadron] A composite particle made of two or more quarks. Includes all mesons and baryons.
\item[HEP] High energy physics. Also known as particle physics.
\item[lepton] Electrons, muons, taus, and their associated neutrinos and antiparticles.
\item[LHC] Large Hadron Collider. A circular proton-proton collider operating at CERN.
\item[meson] A composite particle made of two quarks.
\item[MC] Monte Carlo. A class of algorithms based on random sampling. In high-energy physics, MC events (as opposed to data events) are simulated via MC algorithms, rather than gathered from actual collisions.
\item[model] An HEP model describes which particles exist in nature and how they interact. e.g. the Standard Model.
\item[RNN] Recurrent neural net. A neural architecture described in Section~\ref{sec:RNN}.
\item[search] An attempt to find evidence that collision data is better described by an alternative model than by the Standard Model. e.g. SUSY search.
\item[SM] Standard Model. This model describes how the electromagnetic, weak, and strong forces interact with all known elementary particles, and is described in Chapter~\ref{chap:SM}.
\item[SUSY] Supersymmetry. A class of models described in Chapter~\ref{chap:SUSY}.
\end{symbollist*}