\chapter{Supersymmetry}\label{chap:SUSY}

\section{What Is Supersymmetry?}

\section{Theoretical Support}

\section{How to Find SUSY}

\section{Failing to Find SUSY}

\section{How Long Can This Madness Go On For?}

%%%%%%%%%%%%%%%%%%%%%%%%%%%%%%%%%%%%%%%%%%%%%%%%%%%%%%%%%%%

Supersymmetric (SUSY) theories have been proposed as a potential solution to some of these problems. According to SUSY (Figure~\ref{SUSY}), there is an additional supersymmetric partner to each particle that we currently know of. Each boson has a fermionic superpartner, and each fermion has a bosonic superpartner. This provides a natural solution to the hierarchy problem. Furthermore, if the lightest neutral SUSY particle is prohibited from decaying into a normal-matter state, it could remain undetected and thus form the basis of dark matter. In addition, if SUSY is imposed as a local symmetry, supergravity theories can be formed, merging the standard model and general relativity.

With all of these strong points, it would seem that SUSY is a very promising theory. However, despite our best efforts, attempts to search for SUSY particles have so far proved unfruitful. Clearly, if SUSY is a correct theory, it must undergo spontaneous symmetry breaking which changes the masses of the undiscovered superpartners. Thus it may be that the undiscovered particles are currently outside the energy range of our colliders (though there are upper limits to particle masses before the theory becomes unnatural), or it may be that the particles are too close together in mass, in which case their decay products will be hard to detect. This second possibility will be examined later in the paper, and will form part of the motivation for developing a low-energy particle identification system.