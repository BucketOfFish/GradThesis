\chapter{The History of Understanding the History of Everything}

In the beginning people knew nothing. What is the universe made of? Can you turn a rock into gold? What happens if you get smaller and smaller - is there a whole different world down there? These were just a small subset of questions our early ancestors would never know the answer to, since they all died out before the development of particle physics in the $20^{th}$ century. But let's entertain ourselves by examining a few of the early mental models they had, and how the persistent desire to improve these models led to the development of modern physics.

\section{The Basic Components}

It's somewhat strange at first glance that many ancient cultures had roughly the same classic elements, but upon examination of the things commonly found in nature, we can understand where these groupings came from. Most early building materials, food, and organic compounds came out of and into the earth. These things seemed to cycle endlessly and turn easily into each other, so the element of earth or clay logically served as a catch-all building block for all such materials. Chinese philosophy further divided wood and metal into their own categories separate from earth. Next came fire and water, distinct enough from earth to warrant separate categories of their own. Together, earth, water, and fire also roughly corresponded to the common states of matter (solid, liquid, gas/plasma), so it's understandable that these mental clusterings became common in many early philosophies. The elements of air and aether were somewhat unique to Greek thought however, and were probably later transported into India and folded into the classic Buddhist elements.

In all then, we had air, water, earth, fire, and aether in Greek philosophy, corresponding to the five Platonic solids~\cite{Plato}, and according to Empedocles~\cite{Empedocles}, bound together by love and strife. In Indian philosophy we had air, water, and earth, corresponding to different categories of food~\cite{Chandogya}. This evolved with Greek influence into the five elements of early Buddhism and Hinduism, each with their own associations~\cite{friesian}. In Chinese philosophy we had metal, wood, earth, water, and fire~\cite{Tao}. When combined with air and aether, the system was further expanded to form associations with the sun, moon, and five visible planets~\cite{friesian}.

And so we see how throughout the ancient world, a complex series of inter-related, "common-sense", and entirely wrong memetic systems came to dominate early philosophy. Based on pattern recognition and concept association rather than experimental evidence, these unscientific models guided intellectual pursuits for many centuries.

<The_Sceptical_Chymist>

<chemical revolution>

<John Dalton>

<Einstein brownian motion>

<Mendeleev>

\section{Elements Are Not Elemental}

\section{A New Subatomic World}

\section{Everything Is Particles}