\chapter{The Meaning of It All}

Well, in the end we didn't find evidence of supersymmetry, but the search for knowledge goes on. The techniques we developed here will go into our community's scientific toolkit, and will come in useful for future searches. It's good to know that we're able to continuously come up with new ideas, and to test them rigorously. High energy physics is one of the few scientific fields these days where null results are as publishable and exciting as discoveries, and I personally feel that's a good thing. When it comes to statements about the nature of the universe, or about the fundamental building blocks and forces of reality, we need to test our hypotheses as thoroughly as possible, to make sure we're not jumping to any unsupported conclusions.

To sum up then, here's what we learned in this thesis. First, I talked about a search we did for SUSY, in which I performed analysis on the Z background estimation, and also provided support for sample production and various communal activities. Then I showed the development of two new machine-learning based tools for use in future searches. One of these tools operated on calorimeter data, and the other on tracks. They both related to particle reconstruction, though the calorimeter project was more developed at this stage, and could be used for more purposes. Finally, I talked a bit more about various other projects I have taken on at CERN, in the pursuit of detector and algorithm upgrades which will be beneficial to the entire high energy physics community. These projects included upgrades to vertex and track reconstruction, as well as improvements to detector chips and firmware.

Thank you for reading my thesis, and I hope you found parts of it to be useful or enlightening.