\title{A Search for Supersymmetry with the ATLAS Detector, and the Use of Machine Learning Techniques for Object Classification in High Energy Physics}
\author{Matt Zhang}
\department{Physics}
\schools{B.S. University of Texas at Austin, 2013}
\phdthesis
\advisor{Ben Hooberman}
\degreeyear{2021}
\committee{Professor Mark Neubauer, Chair\\
Associate Professor Ben Hooberman\\
Associate Professor Jessie Shelton\\
Professor Lance Cooper}
\maketitle

\frontmatter

\begin{abstract}
We conduct a search for supersymmetry using data from the ATLAS detector at CERN, in a region with 2 leptons, 2 jets, and large \MET. We also demonstrate the development of various machine learning techniques to enhance similar physics searches in the future, including the use of neural nets on calorimeter data for particle-type classification, particle energy regression, and shower generation.
\end{abstract}

\chapter*{Acknowledgments}

This thesis is for my friends, who almost managed to keep me sane after all these years. I'd also like to thank my advisor Ben Hooberman, for helping me grow into a confident and competent researcher. My gratitude also goes out to my lab mates, and to all the students, professors, and staff of UIUC for creating such a nurturing learning environment. And of course, I'd like to thank the community at CERN for supporting the international particle physics community and allowing this type of research to flourish.