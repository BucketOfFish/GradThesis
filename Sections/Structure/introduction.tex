\chapter*{An Introduction}

If you are reading this thesis, then I have either applied for or have  already received my PhD in Physics from UIUC. If it's the former, thanks for coming to my thesis defense.

I'm going to try to convince you that I know some things about particle physics, and that I've done some things worthy of a doctorate. Hopefully I'll even teach you something new along the way.

Here's a short summary of my thesis, in case you're skimming it over ten minutes before my defense:

\begin{itemize}

    \item Like many particle physicists over the years, I've spent my graduate career searching for evidence of extensions to the Standard Model. My thesis begins by describing the Standard Model and the supersymmetric (SUSY) models which we were hoping to find evidence for\footnote{Spoiler: we found no such evidence.}.
    
    \item After this, I briefly describe the relevant components of the detector with which our precious PhD-sustaining data was collected, and walk through the basic steps and terminology of a particle physics search.
    
    \item Next up is a description of the SUSY search which forms the basis of my thesis work. This is the section which proves that I am actually a physicist as opposed to a computer scientist. I describe the details of our search, and describe my contributions to the project. Results are discussed here.

    \item The focus then shifts to my particular specialty, machine learning. I describe how machine learning works, including the general classes of techniques which I used in my various studies.
    
    \item After that I go through details of two machine learning studies which I led.
    
    \begin{itemize}
    
        \item The first study was on the identification and generation of particle showers in calorimeters. This means that we wrote an algorithm which is able to look at a calorimeter shower after a collision event and both determine what type of particle produced the shower and how much energy the particle had. We could also artificially generate showers for different particles at various energies, a technique which may allow us to circumvent computationally-intensive Monte Carlo (MC) simulations in the future. My main focus was the particle classification section, though I contributed to all parts of the project.

        \item The second study was on the identification of heavy-flavor-decay leptons, which form the largest background in a number of searches. This study used recurrent neural nets (RNN's) to perform lepton classification based on track information.

    \end{itemize}

    \item Following this I describe some detector upgrade related work and other small projects that I have done, including:
    
    \begin{itemize}
        \item work on the Fast TracKer upgrade (FTk)
        \item pixel detector assembly and beam testing work which I did during a one-year residency at Argonne National Lab
        \item improvements to vertex reconstruction algorithms
    \end{itemize}

    \item And of course, no thesis is complete without a conclusion, in which I'll basically tell you all of this information again, except now you'll understand what I'm talking about. I'll also describe methods of applying the machine learning tools I have built to future physics searches.

\end{itemize}

Thanks for reading my thesis. Let's begin.