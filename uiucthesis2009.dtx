\def\fileversion{v2.25a} \def\filedate{2009/10/10}
% \iffalse  % this is a METACOMMENT !
%% Package and Class "uiucthesis2009" for use with LaTeX2e.
%
%<class|package>\NeedsTeXFormat{LaTeX2e}
%<class>\ProvidesClass{uiucthesis2009}
%<package>\ProvidesPackage{uiucthesis2009}
%<class|package>         [\filedate\ \fileversion\ UIUC Thesis (PJC)]
%
%<*driver>
\documentclass{ltxdoc}
\begin{document}
\title{The \textsf{uiucthesis2009} class
       \thanks{This file has version number \fileversion,
       last revised \filedate.}}
\author{Charles Kiyanda\\charles@kiyanda.com\\(Adapted from version 2.25, 2005/03/25 by Peter Czoschke\\Updated in 2007 by Tim Head.)}
\date{\filedate}
\maketitle
\DocInput{uiucthesis2009.dtx}
\end{document}
%</driver>
% \fi
%
% \CheckSum{1164}
%
% \MakeShortVerb{\|}
%
% \def\pkg#1{\textsf{#1}}
% \def\env#1{\textsf{#1}}
%
% \iffalse
%<*example>
% \fi
% \begin{abstract}
% Load the \pkg{uiucthesis2009} class for use with \LaTeX2e
% to produce a document that should conform
% to the format described in
% \emph{Handbook for Graduate Students Preparing to Deposit}\cite{Handbook}. (Actually
% I have checked the requirements from a grad college webpage\cite{thesisweb}.
% I believe 
% this template complies, but there is no guarantee.)
% \end{abstract}
% 
% \section{The User Interface}
%
% This section describes how to use the \pkg{uiucthesis2009} class to
% produce a thesis satisfying the format requirements of the
% Grad College at UIUC.
% I assume that you are familiar with \LaTeX, and highly
% recommend that anyone attempting to use \LaTeX\ to produce a
% thesis have access to a copy of the \LaTeX\ book\cite{Lamport}.
%
% Note that I haven't graduated yet and so I haven't taken this template
% through the ultimate test, which is to actually submit a thesis that ues it.
% I believe this template does conform to the graduate college requirement, but I am,
% \emph{in no way} guaranteeing it conforms to anything.
%
% Also, I'm writing my thesis right now as well. I don't plan to make any more changes to this
% template until I graduate. Hence, if you send me an e-mail right now asking to change some
% detail that you'd think looks better, you're likely to either
% not get a response or receive a somewhat polite ``Wait until january...'' type of response.
% If you firmly believe this template does not conform to the graduate college requirements, be
% very precise and I might look at it. It's just the way it is right now, I'm afraid...
% 
% \subsection{Using \pkg{uiucthesis2009}}
%
% To write a thesis, you load the UIUC thesis definitions
% by loading the \pkg{uiucthesis2009} class at the beginning of
% your \LaTeX\ document with the |\documentclass| command.
% For example,
% \begin{quote} \hrule \begin{verbatim}
\documentclass[edeposit,fullpage]{uiucthesis2009}
% \end{verbatim} \hrule \end{quote}
% \iffalse
%<example>
% \fi
%
% \DescribeMacro{[draftthesis]}
% \DescribeMacro{[fancy]}
% \DescribeMacro{[fullpage]}
% The \pkg{uiucthesis2009} class provides a number of options.
% The |[draftthesis]| option causes each page to have a header
% proclaiming the document to be a draft copy, along with the
% current time and date.
% It also omits the copyright page and prints out any marginal notes
% added with the |\note| macro.
% The |[fancy]| style option produces slightly fancier chapter headings.
% The |[fullpage]| style option makes the margins as small as the
% format requirements allow and uses double-spacing for the text.
% Because wide text columns are generally
% considered harder on the reader
% this is not the default, but is provided as an option because people
% seem to want it.
% The |[fancy]| and |[fullpage]| options are incompatible---choose
% one or the other.
%
% \DescribeMacro{[proquest]}
% The |[proquest]| option is meant to be used when you are ready to deposit your thesis.
% For doctorates, the Grad College requires the submission of a specially
% formatted abstract for the ProQuest publication service. To produce this
% abstract, include the |[proquest]| option and reprocess your file. Everything
% in your \LaTeX\ document will be ignored except the contents of the \env{abstract}
% environment, which are printed out in the format required. Once the option is
% removed from the |\documentclass| command, you can reprocess your thesis as
% normal (the auxiliary files should be intact). To use this option, the name
% of your thesis advisor needs to be specified with the |\advisor| command (see
% below).
%
% \DescribeMacro{[edeposit]}
% Use the |[edeposit]| option if you are depositing your thesis electronically. The title page
% used to be different, but the requirement appears to have been harmonized since. The page 
% numbering is slightly different since the committee
% approval form is not included with your thesis. (The page numbering change is currently the
% only difference of thet edeposit option.  You
% must specify your committee members with the |\committee| command (see below).
%
% \DescribeMacro{[offcenter]}
% The |[offcenter]| option adds 1/2 inch to the left margin of all pages and takes
% away 1/2 inch to the text width, leaving a 1.5in left margin and a 1in right margin.
% I believe this setting satisfies the pre-2009 requirement by the grad college to
% have a 1.5in margin for binding. It should also allow you more room for binding
% if you need it. The new requirement is a minimum 1in margin all around and the 
% |[fullpage]| option without the |[offcenter]| option should achieve this goal.
%
% This can now be done to allow some extra space there for binding purposes, or
% if you use the |[fancy]| option, to allow for more space for the chapter
% numbers at the left side of the page.
% In past versions of \pkg{uiucthesis2009} (the name used was simply uiucthesis)
% , the |[fancy]| option did this by default.
% This version uses symmetric margins by default, even with the
% |[fancy]| option. If you have a lot of chapters (i.e., more than 9), your chapter
% numbers may spill into
% the 1 inch margin required by the Grad College without using this option.
%
% \DescribeMacro{[centerchapter]}
% Normally, the chapter headings are all left-justified on the opening page of
% each chapter. These headings can all be centered by using the |[centerchapter]|
% option for the class. This option is not recommended for use with the |[fancy]|
% option.
%
% \subsection{The Title Page}
%
% The |\maketitle| command is redefined so that it creates a
% title page with the correct format for a thesis at UIUC.
%
% \DescribeMacro{\phdthesis}
% \DescribeMacro{\otherdoctorate}
% \DescribeMacro{\msthesis}
% \DescribeMacro{\othermasters}
% \DescribeMacro{\department}
% \DescribeMacro{\college}
% Use the |\phdthesis| or |\msthesis| to set the correct thesis type.
% If your thesis isn't for a ``Ph.D.'' or ``M.S.'', you can specify
% your degree with either\\
% |\otherdoctorate{|\meta{degree name}|}{|\meta{abbreviation}|}| or\\
% |\othermasters{|\meta{degree name}|}{|\meta{abbreviation}|}|.\\
% For example, specifying |\phdthesis| is equivalent to giving the command\\
% |\otherdoctorate{Doctor of Philosophy}{Ph.D.}|.\\
% The default thesis type is |\phdthesis|. Set your department with\linebreak
% |\department{|\meta{department}|}|. This defines the field your degree
% will be in, so leave out ``Department of.''
% The default department is ``Computer Science''.
% Define your college with |\college{|\meta{college}|}|.
% The default is college is ``Graduate College'';
% you shouldn't need to change it.
%
% \DescribeMacro{\schools}
% Use |\schools{|\meta{school list}|}| to list the previous degrees
% you have received and the schools that you received them from.
% Separate multiple degrees with |\\|.
%
% \DescribeMacro{\degreeyear}
% Use |\degreeyear{|\meta{year}|}| to define the year in which
% you will receive your degree.  The default is the current year.
%
% \DescribeMacro{\advisor}
% \DescribeMacro{\adviser}
% Use |\advisor{|\meta{advisor name}|}| or |\adviser{|\meta{advisor name}|}|
% to specify the name of your
% advisor. This is needed to produce the ProQuest abstract
% (see the |[proquest]| option above). You only need to submit a
% ProQuest abstract if you are a doctoral candidate.
%
% \DescribeMacro{\committee}
% Use |\committee{|\meta{committee members}|}| to specify the members
% of your committee and their titles as you want them to appear on the
% title page. Separate members with |\\|. This is needed for
% all forms of thesis submission. To respect the graduate college guidelines,
% you must use the full title of each committee members. The committee chair should
% appear first with the designation ``,chair''. Your thesis adviser should appear second
% with the title ``,Director of Research''. See the graduate college website for details.
%
% \DescribeMacro{\volume}
% The |\volume| macro provides nominal support for very long theses that must
% be broken up into multiple volumes. Use |\volume{|\meta{number}|}|
% to specify the volume number (a single arabic numeral). All this macro
% does is place the word VOLUME with the number you specify on the title
% page. You have to take care of what appears in each volume. The easiest
% way to do this is to create two separate source files, one for each
% volume.
%
% Here's how to produce an example similar to that in \cite{Handbook}.
% \begin{quote} \hrule \begin{verbatim}
\begin{document}

\title{Coffee Consumption of Graduate Students \\
       Trying to Finish Dissertations}
\author{Juan Valdez}
\department{Food Science}
\schools{B.A., University of Columbia, 1981\\
         A.M., University of Illinois at Urbana-Champaign, 1986}
\phdthesis
\advisor{Java Jack}
\degreeyear{1994}
\committee{Professor Prof Uno, Chair\\Professor Prof Dos, Director of Research\\Assistant Professor Prof Tres\\Adjunct Professor Prof Quatro}
\maketitle
% \end{verbatim} \hrule \end{quote}
% \iffalse
%<example>
% \fi
%
% \subsection{Front Matter}
%
% \DescribeMacro{\frontmatter}
% Typically, a thesis might have an Abstract, a Dedication, some
% Acknowledgments, and a Preface before the Table of Contents.
% Use the |\frontmatter| command to start this preliminary section
% of the thesis.
% The |\frontmatter| command sets the page number of the next page
% to roman numeral iii (or ii if the |[edeposit]| option is used).
% (The title page is page i, and the certificate
% of committee approval, the ``red-bordered form,'' is page ii.)
%
% \DescribeEnv{abstract}
% The abstract should appear in the \env{abstract} environment. Normally,
% this just produces another chapter with |\chapter*{\abstractname}|,
% where |\abstractname| is ``Abstract'' (see User Customization below),
% but if the |[proquest]| option
% is specified, then the contents of this environment are used
% for the ProQuest abstract.
%
% \DescribeEnv{dedication}
% A dedication page can be printed with the \env{dedication} environment.
% This produces a separate page with the dedication centered horizontally
% and vertically, with the text in italics.
%
% After this front matter comes the Table of Contents,
% List of Tables, List of Figures, etc.  Use the standard \LaTeX\
% commands |\tableofcontents|, |\listoftables|, |\listoffigures|, etc.,
% to generate them.
% In the \pkg{uiucthesis2009} format these lists are all single spaced.
%
% \DescribeEnv{symbollist}
% \DescribeEnv{symbollist*}
% Optionally, these tables can be followed by a List of Abbreviations and/or
% List of Symbols. Introduce these with the |\chapter| command. To aid in
% making these lists, the \env{symbollist} and \env{symbollist*} environments are
% defined in \pkg{uiucthesis2009}. These environments produce a two-column list
% as illustrated below. By default the left column is 1 inch wide but can
% be specified with an optional argument. In the starred environment, the left
% column is left-justified, otherwise it is centered. See the example below.
%
% Here's an example of what the front matter of a typical
% thesis looks like.  First comes the Abstract and the Dedication, both of
% which are optional.
% \begin{quote} \hrule \begin{verbatim}
\frontmatter

%% Create an abstract that can also be used for the ProQuest abstract.
%% Note that ProQuest truncates their abstracts at 350 words.
\begin{abstract}
This is a comprehensive study of caffeine consumption by graduate
students at the University of Illinois who are in the very final
stages of completing their doctoral degrees. A study group of six
hundred doctoral students\ldots.
\end{abstract}

%% Create a dedication in italics with no heading, centered vertically
%% on the page.
\begin{dedication}
To Father and Mother.
\end{dedication}

%% Create an Acknowledgements page, many departments require you to
%% include funding support in this.
\chapter*{Acknowledgments}

This project would not have been possible without the support of
many people. Many thanks to my adviser, Lawrence T. Strongarm, who
read my numerous revisions and helped make some sense of the
confusion. Also thanks to my committee members, Reginald Bottoms,
Karin Vegas, and Cindy Willy, who offered guidance and support.
Thanks to the University of Illinois Graduate College for awarding
me a Dissertation Completion Fellowship, providing me with the
financial means to complete this project. And finally, thanks to
my husband, parents, and numerous friends who endured this long
process with me, always offering support and love.

%% The thesis format requires the Table of Contents to come
%% before any other major sections, all of these sections after
%% the Table of Contents must be listed therein (i.e., use \chapter,
%% not \chapter*).  Common sections to have between the Table of
%% Contents and the main text are:
%%
%% List of Tables
%% List of Figures
%% List Symbols and/or Abbreviations
%% etc.

\tableofcontents
\listoftables
\listoffigures
% \iffalse
%<example>
% \fi
% \end{verbatim} \hrule \end{quote}
%
% If you want a List of Symbols or Abbreviations, you can do so as follows:
% \begin{quote} \hrule \begin{verbatim}
%% Create a List of Abbreviations. The left column
%% is 1 inch wide and left-justified
\chapter{List of Abbreviations}

\begin{symbollist*}
\item[CA] Caffeine Addict.
\item[CD] Coffee Drinker.
\end{symbollist*}

%% Create a List of Symbols. The left column
%% is 0.7 inch wide and centered
\chapter{List of Symbols}

\begin{symbollist}[0.7in]
\item[$\tau$] Time taken to drink one cup of coffee.
\item[$\mu$g] Micrograms (of caffeine, generally).
\end{symbollist}
% \end{verbatim} \hrule \end{quote}
% \iffalse
%<example>
% \fi
%
% \subsection{Main Matter}
%
% \DescribeMacro{\mainmatter}
% Begin the main body of your thesis with the |\mainmatter| command.
% It resets the page number to arabic numeral 1.
% You can now use any of the commands defined by the
% the book document class to write your thesis.
%
% In the following example, each of the chapters
% has been broken out into separate files that are inserted into
% this main file with the |\include| command.  This allows the
% thesis to be proofed quickly while it is being revised with
% the |\includeonly| command.  To provide an example of what the
% chapter headings look like, one chapter has been explicitly
% coded. (Try recompiling the example file with the |[fancy]|
% option instead of |[fullpage]| to see the effect.)
%
% \begin{quote} \hrule \begin{verbatim}
\mainmatter
% Sample chapter to test margins
\chapter{This world}
\section{Of the Nature of Flatland}


I call our world Flatland, not because we call it so, but to make its
nature clearer to you, my happy readers, who are privileged to live in
Space.

Imagine a vast sheet of paper on which straight Lines, Triangles,
Squares, Pentagons, Hexagons, and other figures, instead of remaining
fixed in their places, move freely about, on or in the surface, but
without the power of rising above or sinking below it, very much like
shadows--only hard with luminous edges--and you will then have a pretty
correct notion of my country and countrymen.  Alas, a few years ago, I
should have said "my universe:"  but now my mind has been opened to
higher views of things.

In such a country, you will perceive at once that it is impossible that
there should be anything of what you call a "solid" kind; but I dare
say you will suppose that we could at least distinguish by sight the
Triangles, Squares, and other figures, moving about as I have described
them.  On the contrary, we could see nothing of the kind, not at least
so as to distinguish one figure from another.  Nothing was visible, nor
could be visible, to us, except Straight Lines; and the necessity of
this I will speedily demonstrate.

Place a penny on the middle of one of your tables in Space; and leaning
over it, look down upon it.  It will appear a circle.

But now, drawing back to the edge of the table, gradually lower your
eye (thus bringing yourself more and more into the condition of the
inhabitants of Flatland), and you will find the penny becoming more and
more oval to your view, and at last when you have placed your eye
exactly on the edge of the table (so that you are, as it were, actually
a Flatlander) the penny will then have ceased to appear oval at all,
and will have become, so far as you can see, a straight line.

The same thing would happen if you were to treat in the same way a
Triangle, or a Square, or any other figure cut out from pasteboard.  As
soon as you look at it with your eye on the edge of the table, you will
find that it ceases to appear to you as a figure, and that it becomes
in appearance a straight line.  Take for example an equilateral
Triangle--who represents with us a Tradesman of the respectable class.
Figure 1 represents the Tradesman as you would see him while you were
bending over him from above; figures 2 and 3 represent the Tradesman,
as you would see him if your eye were close to the level, or all but on
the level of the table; and if your eye were quite on the level of the
table (and that is how we see him in Flatland) you would see nothing
but a straight line.

When I was in Spaceland I heard that your sailors have very similar
experiences while they traverse your seas and discern some distant
island or coast lying on the horizon.  The far-off land may have bays,
forelands, angles in and out to any number and extent; yet at a
distance you see none of these (unless indeed your sun shines bright
upon them revealing the projections and retirements by means of light
and shade), nothing but a grey unbroken line upon the water.

Well, that is just what we see when one of our triangular or other
acquaintances comes towards us in Flatland.  As there is neither sun
with us, nor any light of such a kind as to make shadows, we have none
of the helps to the sight that you have in Spaceland.  If our friend
comes closer to us we see his line becomes larger; if he leaves us it
becomes smaller; but still he looks like a straight line; be he a
Triangle, Square, Pentagon, Hexagon, Circle, what you will--a straight
Line he looks and nothing else.

You may perhaps ask how under these disadvantagous circumstances we are
able to distinguish our friends from one another: but the answer to
this very natural question will be more fitly and easily given when I
come to describe the inhabitants of Flatland.  For the present let me
defer this subject, and say a word or two about the climate and houses
in our country.


\include{1-introduction}
\include{2-related}
\include{3-model}
\include{4-predictions}

\chapter{Conclusions}

We conclude that graduate students like coffee.
% \iffalse
%<example>
% \fi
% \end{verbatim} \hrule \end{quote}
%
% \subsection{Reference Matter}
%
% \DescribeMacro{\appendix}
% To switch from the body of your thesis to the reference material
% at the end, you should use the standard \LaTeX\ |\appendix| command.
% In \pkg{uiucthesis2009}, there is also a starred version of this command
% that eliminates the lettering of the appendices (use if you have
% a single appendix). Note that if you use |\appendix*| along with
% the |[fancy]| option, you may want to put ``Appendix:'' at
% the beginning of the chapter title.
%
%
% \begin{quote} \hrule \begin{verbatim}
\appendix*

\include{Appendix.tex}
% \end{verbatim} \hrule \end{quote}
% \iffalse
%<example>
% \fi
%
% \subsection{Back Matter}
%
% \DescribeMacro{\backmatter}
% The last few chapters in your thesis should not have chapter
% numbers, but should be listed in the Table of Contents.
% These chapters include the Bibliography, the Index,
% and the Vita.  \LaTeX's |\backmatter| command accomplishes this.
%
% \DescribeMacro{\bibliography}
% Use the standard \LaTeX\ bibliography commands to
% create your bibliography. Most people will use Bib\TeX\ to do this.
% (See \cite{Lamport}). For those in the sciences, you may want to
% check out the \pkg{cite} package (it's pretty standard), which
% will produce numerical citations that are sorted and compressed.
% You can also use the \pkg{natbib} package. Both of these packages
% can do either bracketed citations or superscript citations.
%
% \DescribeMacro{\vita}
% The |\vita| command begins a new chapter for your vita.
% In fact, it does exactly the same thing as |\chapter{\vitaname}|,
% where |\vitaname| is ``Vita.''
%
% \begin{quote} \hrule \begin{verbatim}
\backmatter

\bibliography{thesisbib}

\chapter{Vita}

Juan Valdez was born\ldots.

\end{document}
% \end{verbatim} \hrule \end{quote}
%
% \iffalse
%</example>
% \fi
%
% \section{User Customization}
%
% \DescribeMacro{\draftheader}
% If you don't like the header that the the |[draftthesis]| option
% creates, you can redefine the |\draftheader| command so that it
% produces whatever text you want.
%
% \DescribeMacro{\thesisspacing}
% The \pkg{uiucthesis2009} class loads \pkg{setspace} for the
% line spacing commands.  See the documentation in that package
% for more information on the commands it provides.
% By default, \pkg{uiucthesis2009} uses one and a half line spacing,
% or double spacing if the |[fullpage]| option is specified.
% If you're unhappy with that, you can override it by redefining the
% |\thesisspacing| command.
%
% \DescribeMacro{\nocopyrightpage}
% Unless the |[draftthesis]| option is used, a page with the copyright notice
% is printed before the title page. If you don't want this page to
% appear, even in the final version, put the |\nocopyrightpage| macro
% somewhere in the preamble.
%
% \DescribeMacro{\toclabels}
% Some departments require the Table of Contents, List of Tables and
% List of Figures to have a ``Page'' heading over the page numbers
% on the first page. This can be accomplished by putting the |\toclabels|
% command somewhere before the |\tableofcontents| command. (NOTE: if you put
% the |\tableofcontents| command in a separate file that you |\include| in
% the main file, the |\toclabels| command must also be in that file.)
%
% \DescribeMacro{\chaptertitlefont}
% \DescribeMacro{\sectiontitlefont}
% \DescribeMacro{\subsectiontitlefont}
% \DescribeMacro{\subsubsectiontitlefont}
% These macros contain the font declarations for the corresponding sectioning
% levels and can be redefined to suit your aesthetic desires. Use
% |\renewcommand| to do this in the preamble.
%
% \DescribeMacro{\chapternumberfont}
% The |\chapternumberfont| macro is really most applicable when the |[fancy]| option is used. It
% specifies the font declaration used for the chapter number set in the left
% margin next to the title. Otherwise it specifies the font used to print
% the words ``Chapter \#'' at the top of each chapter's opening page. Use
% |\renewcommand| to redefine this macro in the preamble.
%
% \DescribeMacro{\chaptertitleheight}
% |\chaptertitleheight| is the amount of space allotted for the chapter title at the top
% of the page. You can redefine it using |\setlength| in the preamble.
%
% \DescribeMacro{\bibname}
% |\bibname| is a standard \LaTeX\ macro that contains the title of the reference
% section at the end of your thesis; ``References'' by default. Use
% |\renewcommand| to redefine it in the preamble.
%
% \DescribeMacro{\vitaname}
% Like |\bibname| but it contains the name used for your vita at the very end.
% The Grad College allows ``Vita'', ``Author's Biography'', or ``Curriculum Vitae'',
% each of which is slightly different in format.  See \cite{Handbook}.
%
% \DescribeMacro{\abstractname}
% Like above, but for the abstract. By default, ``Abstract''.
%
% \DescribeMacro{other "name"s}
% Similarly, the titles for the Table of Contents, List of Figures, and List of Tables
% are stored in the macros |\contentsname|, |\listfigurename|, and |\listtablename|,
% respectively. Their default values are the names in the previous sentence. There are
% also the macros |\chaptername|, |\appendixname|, |\indexname|, |\partname|,
% |\tablename|, and |\figurename| that contain the appellations for chapter and appendix
% headings (not applicable with the |[fancy]| option), the index, parts, tables, and figures.
% Their default values are Chapter, Appendix, Index, Part, Table, and Figure, respectively.
% All of these macros can
% be redefined in the preamble with |\renewcommand|.  These macros are all part of
% the standard \LaTeX\ formalism, they are just included here for the reader's convenience.
% For example, some departments require chapter headings to be in all caps, which can
% be done by changing the |\chaptername| macro to be CHAPTER.
%
% \DescribeMacro{\note}
% This command inserts a marginal note just like |\marginpar| with two distinctions:
% First, the note is single-spaced in a smaller type for compactness.
% Second, it is only printed when
% the |[draftthesis]| option is used. Since marginal notes are not allowed in the final
% draft, the |\note| command is recommended over the |\marginpar| command.
%
% \section{Backwards Compatibility}
%
% Compatibility with previous versions of \pkg{uiucthesis2009} are supported. Previously it
% was implemented as a package rather than a class, in which case you opened the document
% with:
% \begin{quote} \hrule \begin{verbatim}
% \documentclass[oneside,...]{book}
% \usepackage[...]{uiucthesis2009}
% \end{verbatim} \hrule \end{quote}
% To provide backwards-compatibility, a style file is provided that has the same functionality
% of the class file described herein. Similarly, (really) old versions of \pkg{uiucthesis2009}
% used the \env{preliminary} and \env{thesis} environments, which are now deprecated
% but backwards-compatibility support is still provided.
%
% \section{Other Issues}
%
% \subsection{Paper Size and PDF Files}
%
% The default paper size for most \LaTeX\ distributions is A4, which is slightly different
% the 8.5 x 11 size for letter paper in the U.S\@. The Graduate College requires theses
% to be letter paper size.  In addition, many departments want soft copies of the thesis
% in PDF format. Unfortunately, the program used to convert the dvi file to PDF format
% (|dvipdfm| on my computer, which is a Windows machine with Mik\TeX installed on it)
% often produces PDF files in A4 format, even if |letterpaper|
% is specifically specified in the your \TeX source file. If you're having this problem,
% either run the PDF conversion utility from the command line with the right flag ---
% for example, |dvipdfm -p letter| on my computer --- or change the default paper size in
% the config file for your PDF conversion utility, which will then fix this problem
% permanently.  On my computer, this can be done by going to the |dvipdfm\config|
% subdirectory off of the main \TeX installation directory (usually |C:\texmf|). In this
% directory is a file called |config| that has a line for the default paper size.
%
% \subsection{Reference Lists at the Chapter Level}
%
% The \pkg{cite} package includes a style file \pkg{chapterbib.sty} that can be used
% to do a list of references for each chapter instead of just one big list at the
% end of the thesis.  I've not used this style before so you're on your own if you
% want to do this, but I think it is rather straightforward\ldots.
%
% \StopEventually{%
%   \begin{thebibliography}{9}
%     \bibitem{Handbook}
%     \emph{Handbook for Graduate Students Preparing to Deposit}.
%     \newblock Graduate College, University of Illinois at
%     Urbana-Champaign, 2004
%
%     \bibitem{thesisweb}
%      \emph{Grad College webpage with thesis requirements}.
%      \newblock |http://www.grad.illinois.edu/graduate-college-thesis-requirements|
%
%     \bibitem{Lamport}
%     Leslie Lamport.
%     \newblock \emph{\LaTeX: A Document Preparation System}.
%     \newblock Addison-Wesley, 1994.
%   \end{thebibliography}
% }
%
% \section{Implementation}
%
% This section shows the implementation of the \pkg{uiucthesis2009} class.
% Unless you are interested in the details of how \pkg{uiucthesis2009} works,
% you probably don't need to read it.
%
% \iffalse
%<*class|package>
\RequirePackage{setspace}
% \fi
%
% \subsection{Compatibility}
%
% Provide compatibililty with older versions of \LaTeX.
% \begin{macro}{\@ifundefined}
%    \begin{macrocode}
\expandafter\ifx\csname @ifundefined\endcsname\relax
  \def\@ifundefined#1{%
    \expandafter\ifx\csname#1\endcsname\relax
      \expandafter\@firstoftwo
    \else
      \expandafter\@secondoftwo
    \fi}
\fi
%    \end{macrocode}
% \end{macro}
%
% \begin{macro}{\MakeUppercase}
%    \begin{macrocode}
\@ifundefined{MakeUppercase}{\let\MakeUppercase=\uppercase}{}
%    \end{macrocode}
% \end{macro}
%
% \subsection{Option Processing}
%
%    \begin{macrocode}
\newif\if@thesisdraft \@thesisdraftfalse
\newif\if@thesisfancy \@thesisfancyfalse
\newif\if@fullpage \@fullpagefalse
\newif\if@largecaps \@largecapsfalse
\newif\if@proquest \@proquestfalse
\newif\if@edeposit \@edepositfalse
\newif\if@thesisoffcenter \@thesisoffcenterfalse
\newif\if@centerchapter \@centerchapterfalse
%    \end{macrocode}
%
%    \begin{macrocode}
\DeclareOption{draftthesis}{\@thesisdrafttrue}
\DeclareOption{fancy}{\@thesisfancytrue}
\DeclareOption{fullpage}{\@fullpagetrue}
\DeclareOption{proquest}{\@proquesttrue}
\DeclareOption{toclabels}{\AtBeginDocument{\toclabels}}
\DeclareOption{edeposit}{\@edeposittrue}
\DeclareOption{offcenter}{\@thesisoffcentertrue}
\DeclareOption{centerchapter}{\@centerchaptertrue}
%    \end{macrocode}
%
% The |[largecaps]| option causes the title and author's name to
% be use a ``large caps'' font on the title page.  Otherwise,
% \pkg{uiucthesis2009} just converts them to uppercase and uses the
% normal fonts.  The difference is that the spacing between the
% characters in the large caps font is tuned for setting type in all caps.
%
% The large caps font is \emph{not a standard font}, and so it will not exist
% unless you have installed it.
%
%    \begin{macrocode}
\DeclareOption{largecaps}{\@largecapstrue}
%    \end{macrocode}
%
% Load the \pkg{book} class with the |[oneside]| and |[letterpaper]| options
%
%    \begin{macrocode}
%<class>\DeclareOption*{\PassOptionsToClass{\CurrentOption}{book}}
%<class>\PassOptionsToClass{letterpaper,oneside}{book}
\ProcessOptions
%<class>\LoadClass{book}
%    \end{macrocode}
%
% If the |[proquest]| option is used, turn off output to auxiliary files so
% that the thesis doesn't have to be recompiled again to get all the references
% correct. Also double-space the ProQuest abstract and use the full page.
%
%    \begin{macrocode}
\if@proquest
    \nofiles    % don't overwrite the .aux files
    \def\makeindex{}
    \@thesisfancyfalse
    \@fullpagetrue
\fi
%    \end{macrocode}
%
% If the |[draftthesis]| option was specified, define the |\draftheader| macro.
%
%    \begin{macrocode}
\if@thesisdraft
  \newcount\timehh\newcount\timemm
  \timehh=\time \divide\timehh by 60
  \timemm=\time \count255=\timehh \multiply\count255 by -60
    \advance\timemm by \count255
  \def\draftheader{\slshape Draft of \today\ at
  \ifnum\timehh<10 0\fi\number\timehh\,:\,\ifnum\timemm<10 0\fi\number\timemm}%
\fi
%    \end{macrocode}
%
% Define the |\toclabels| command which prints the headings in the Table of Contents,
% List of Figures and List of Tables.
%
%    \begin{macrocode}
\newcommand{\toclabels}{%
    \addtocontents{toc}{\vspace*{-\baselineskip}\hfill Page\endgraf}%
    \addtocontents{lof}{\vspace*{-\baselineskip}~Figure\hfill Page\endgraf}%
    \addtocontents{lot}{\vspace*{-\baselineskip}~Table\hfill Page\endgraf}}
%    \end{macrocode}
%
% \subsection{Title Page}
%
% \begin{macro}{\title}
% \begin{macro}{\author}
% Override the standard definitions of |\title| and |\author| to also
% define uppercased versions.
%    \begin{macrocode}
\def\@mkuptitle#1{\gdef\@Utitle{#1}}
\def\title#1{\gdef\@title{#1}\MakeUppercase{\protect\@mkuptitle{#1}}}
\def\@mkupauthor#1{\gdef\@Uauthor{#1}}
\def\author#1{\gdef\@author{#1}\MakeUppercase{\protect\@mkupauthor{#1}}}
%    \end{macrocode}
% \end{macro}
% \end{macro}
%
% \begin{macro}{\phdthesis}
% \begin{macro}{\msthesis}
% \begin{macro}{\otherdoctorate}
% \begin{macro}{\othermasters}
% \begin{macro}{\department}
% \begin{macro}{\college}
% \begin{macro}{\schools}
% \begin{macro}{\degreeyear}
% \begin{macro}{\committee}
% \begin{macro}{\volume}
% Macros to set title page elements.
%    \begin{macrocode}
\def\phdthesis{\def\@degree{Doctor of Philosophy}
    \def\degree{Ph.D.}
    \def\@thesisname{DISSERTATION}
    \def\@committeename{Doctoral Committee:}
    }
\def\msthesis{\def\@degree{Master of Science}
    \def\degree{M.S.}
    \def\@thesisname{THESIS}
    \def\@committeename{Master's Committee:}
    }
\newcommand{\otherdoctorate}[2]{\def\@degree{#1}
    \def\degree{#2}
    \def\@thesisname{DISSERTATION}
    \def\@committeename{Doctoral Committee:}
    }
\newcommand{\othermasters}[2]{\def\@degree{#1}
    \def\degree{#2}
    \def\@thesisname{THESIS}
    \def\@committeename{Master's Committee:}
    }
\def\department#1{\def\@dept{#1}}
\def\college#1{\def\@college{#1}}
\def\schools#1{\def\@schools{#1}}
\def\degreeyear#1{\def\@degreeyear{#1}}
\newcommand{\committee}[1]{\gdef\@committee{#1}}
\newcommand*{\volume}[1]{\gdef\thesis@volume{VOLUME~#1}}
\newcommand*{\thesis@volume}{}
\if@edeposit
  \gdef\@committee{%
%<class>    \ClassError{uiucthesis2009}{A committee must be specified for e-deposit dissertations.}%
%<package>    \PackageError{uiucthesis2009}{A committee must be specified for e-deposit dissertations.}%
    {Use \protect\committee\space with members separated by \protect\\'s.}}
\fi
%    \end{macrocode}
% \end{macro}
% \end{macro}
% \end{macro}
% \end{macro}
% \end{macro}
% \end{macro}
% \end{macro}
% \end{macro}
% \end{macro}
% \end{macro}
%
% \begin{macro}{\copyrightnotice}
% Define the copyright notice as a macro so that the user
% can change it if desired.
%    \begin{macrocode}
\def\copyrightnotice{\copyright~\@degreeyear~by \@author. All rights reserved.}
%    \end{macrocode}
% \end{macro}
% \begin{macro}{\nocopyrightpage}
% The printing of the copyright page can also be turned off with the
% |\nocopyrightpage| command (must come before |\maketitle|):
%    \begin{macrocode}
\newif\if@thesiscrpage \@thesiscrpagetrue
\let\nocopyrightpage\@thesiscrpagefalse
\if@thesisdraft\nocopyrightpage\fi
%    \end{macrocode}
% \end{macro}
%
% Set the default title page elements.
%    \begin{macrocode}
\phdthesis
\department{Computer Science}
\college{Graduate College}
\def\@schools{}
\def\@degreeyear{\number\year}
%    \end{macrocode}
%
%
% \begin{macro}{\maketitle}
% Redefine \pkg{book}'s |\maketitle| command to produce
% the titlepage in the correct format.
%
%    \begin{macrocode}
\renewcommand\maketitle{
%    \end{macrocode}
% Print the copyright page if we're supposed to.
%    \begin{macrocode}
    \if@thesiscrpage
        \newpage
        \thispagestyle{empty}
        \null\vfill
        \centerline{\copyrightnotice}%
        \vskip 3ex % skip to visually center copyright notice
        \vfill
    \fi
%    \end{macrocode}
% Now start a new page for the title page.  It is single-spaced.
%    \begin{macrocode}
    \newpage
    \thispagestyle{empty}%
    \enlargethispage{1in}%
    \begingroup
    \def\baselinestretch{1}
%    \end{macrocode}
% Check what size font we are using for the text and select a
% smaller size appropriately.
%    \begin{macrocode}
    \ifnum \@ptsize=2
        \@normalsize
        \newcommand{\thesis@small}{\small}
    \else
        \large
        \newcommand{\thesis@small}{\@normalsize}
    \fi
%    \end{macrocode}
% We have to be careful to get the vertical position right.  The
% easiest way to do this seems to be to just set |\topmargin|,
% |\headheight|, and |\headsep| for this page.
%    \begin{macrocode}
    \headheight=0pt \headsep=0pt
    \topmargin=0in
%    \end{macrocode}
% Adjust the horizontal spacing so that the title page
% is centered on the page even if the rest of the document isn't.
% I'm not sure when |\textwidth| changes take place, so instead
% we calculate the correct |\oddsidemargin| to center the text column.
%    \begin{macrocode}
    \@tempdima=\paperwidth
    \advance\@tempdima by -\textwidth
    \divide\@tempdima by 2
    \advance\@tempdima by -1in
    \oddsidemargin=\@tempdima
    \let\evensidemargin=\oddsidemargin

%    \end{macrocode}
% Create the title page. Different spacing is used depending on whether the
% |[edeposit]| option is specified. Include the committee and the paragraph
% at the bottom of the page for e-deposit theses, as required by the Grad College.
%    \begin{macrocode}
    \newdimen\thesis@dim
    \if@edeposit
        \thesis@dim=1.5in
    \else
        \thesis@dim=1.5in
    \fi
    \newdimen\ct@dim
    \newdimen\cn@dim
    \ct@dim=\oddsidemargin
    \advance\ct@dim by -0.3125in
    \cn@dim=\oddsidemargin
    \advance\cn@dim by -0.6875in
    \if@largecaps
        \def\lc@selectfont{\fontshape{lc}\selectfont}%
    \else
        \def\lc@selectfont{}%
    \fi
    \begin{center}
    \if@edeposit
        \vbox to 1in{}
    \else
        \vbox to 1in{}
    \fi
    \vbox to \thesis@dim{%
        {\lc@selectfont\@Utitle}
        \if@thesisdraft
        \\[12pt]
        \draftheader
        \fi
        \vfil}%
    \vbox to 1.5in{%
        {\lc@selectfont BY}\\[12pt]
        {\lc@selectfont\@Uauthor}\\[12pt]
        \vfil}%
    \vbox to 0.5in{\thesis@volume\vfil}
    \vbox to 2.0in{%
        {\lc@selectfont \@thesisname}\\[12pt]
        Submitted in partial fulfillment of the requirements\\
        for the degree of \@degree\ in \@dept\\
        in the \@college\ of the\\
        University of Illinois at Urbana-Champaign, \@degreeyear\vfil}
	 \vskip -2ex
	 \vbox to 0.35in{
	 Urbana, Illinois}
	 \end{center}
	\begin{flushleft}
        \vbox to 0.3in{
        \hspace{-\ct@dim}\@committeename\\}
        \hspace{-\cn@dim}\begin{tabular}{l}\@committee\end{tabular}\vfil
	\end{flushleft}
    \newpage
    \endgroup
}
%    \end{macrocode}
% \end{macro}
%
% \subsection{Front Matter}
%
% \begin{macro}{\frontmatter}
% Redefine |\frontmatter| so that it sets the page number to 2 or 3, depending on whether or
% not the |[edeposit]| option is given.
%    \begin{macrocode}
\let\thesis@frontmatter=\frontmatter
\def\frontmatter{%
    \thesis@frontmatter
    \if@edeposit
        \setcounter{page}{2}
    \else
        \setcounter{page}{3}
    \fi}
%    \end{macrocode}
% \end{macro}
%
% \subsection{Table of Contents}
%
% \begin{macro}{\contentsname}
% Use ``Table of Contents'' instead of ``Contents''.
%    \begin{macrocode}
\renewcommand\contentsname{Table of Contents}
%    \end{macrocode}
% \end{macro}
%
% \begin{macro}{\l@chapter}
%    This code is a modified version of the code in the 1996/05/26 release
%    of classes.dtx that produces leader dots between the chapter
%    name and the page number.
%
%    This macro formats the entries in the table of contents for
%    chapters. It is very similar to |\l@part|
%
%    First we make sure that if a pagebreak should occur, it occurs
%    \emph{before} this entry. Also a little whitespace is added and a
%    group begun to keep changes local.
%    \begin{macrocode}
\renewcommand*\l@chapter[2]{%
  \ifnum \c@tocdepth >\m@ne
    \addpenalty{-\@highpenalty}%
    \vskip 1.0em \@plus 0.2em \@minus 0.2em
%    \end{macrocode}
%
%    The macro |\numberline| requires that the width of the box that
%    holds the part number is stored in \LaTeX's scratch register
%    |\@tempdima|. Therefore we put it there. We begin a group, and
%    change some of the paragraph parameters.  These are different
%    from the defaults for the standard report or book class.
%    \begin{macrocode}
    \setlength\@tempdima{1.5em}
    \begingroup
      \leftskip \z@ \rightskip \@tocrmarg \parfillskip -\rightskip
      \parindent \z@
%    \end{macrocode}
%    Then we leave vertical mode and switch to a bold font.
%    \begin{macrocode}
      \leavevmode \bfseries
%    \end{macrocode}
%    Because we do not use |\numberline| here, we have do some fine
%    tuning `by hand', before we can set the entry. We discourage but
%    not disallow a pagebreak immediately after a chapter entry.
%    We use leaders between the chapter title and the page number,
%    unlike the standard report or book class.
%    \begin{macrocode}
      \advance\leftskip\@tempdima
      \hskip -\leftskip
      #1\nobreak
      \leaders\hbox{$\m@th\mkern\@dotsep mu\hbox{.}\mkern\@dotsep mu$}
      \hfil \nobreak\hbox to\@pnumwidth{\hss #2}\par
      \penalty\@highpenalty
    \endgroup
  \fi}
%    \end{macrocode}
% \end{macro}
%
% \begin{macro}{\tableofcontents}
% We want the Table of Contents to be single-spaced, so
% we save the original definition, and then arrange it so
% that the new |\tableofcontents| calls |\singlespacing| before calling
% the original definition. Then set the flag mentioned above.
%    \begin{macrocode}
\let\thesis@tableofcontents=\tableofcontents
\def\tableofcontents{{\singlespacing\thesis@tableofcontents}}
%    \end{macrocode}
% \end{macro}
%
% \begin{macro}{\listoftables}
% \begin{macro}{\listoffigures}
% Similarly, redefine |\listoftables| and |\listoffigures| so
% that they use single spacing.
%    \begin{macrocode}
\let\thesis@listoftables=\listoftables
\def\listoftables{\newpage%
    \addcontentsline{toc}{chapter}{\listtablename}%
    {\singlespacing\thesis@listoftables}}
\let\thesis@listoffigures=\listoffigures
\def\listoffigures{\newpage%
    \addcontentsline{toc}{chapter}{\listfigurename}%
    {\singlespacing\thesis@listoffigures}}
%    \end{macrocode}
% \end{macro}
% \end{macro}
%
% \subsection{Other Frontmatter}
%
% \begin{environment}{abstract}
% The \env{abstract} environment is special because its contents are also
% used for the ProQuest abstract, which we need the advisor's name for:
% \begin{macro}{\adviser}
% \begin{macro}{\advisor}
% Two versions of this macro are provided due to the ambiguity of the spelling
% of the word ``adviser''.
%    \begin{macrocode}
\newcommand*{\advisor}[1]{\gdef\@advisor{#1}}
\newcommand*{\adviser}[1]{\gdef\@advisor{#1}}
%    \end{macrocode}
% \end{macro}
% \end{macro}
% If the |[proquest]| option was specified, erase the definition for |\maketitle|
% since we don't want a title page, and print an error if the advisor's name is
% not specified. Then define the abstract environment to create the ProQuest abstract
% and then end the document.
%    \begin{macrocode}
\def\abstractname{Abstract}
\if@proquest
 \def\maketitle{}
 \def\@advisor{%
%<class>    \ClassError{uiucthesis2009}{An advisor must be specified for the ProQuest abstract}%
%<package>    \PackageError{uiucthesis2009}{An advisor must be specified for the ProQuest abstract}%
    {Use \protect\advisor\space to specify a name}}
 \newenvironment{abstract}{%
    \newpage
    \pagestyle{empty}
    \setcounter{page}{1}
    \begin{singlespace}\begin{center}
    \@Utitle\\[\baselineskip]
    \@author, \degree\\
    Department of \@dept\\
    University of Illinois at Urbana-Champaign, \@degreeyear\\
    \@advisor, Adviser\\[\baselineskip]
    \end{center}\end{singlespace}\par\noindent\ignorespaces
    }{
    \newpage
    \aftergroup\enddocument
    \aftergroup\endinput
    }
%    \end{macrocode}
% If we are doing normal processing (no |[proquest]| option), simply define
% the \env{abstract} environment to start a regular chapter.
%    \begin{macrocode}
\else
 \newenvironment{abstract}{\chapter*{\abstractname}}{}
\fi
%    \end{macrocode}
% \end{environment}
%
% \begin{environment}{dedication}
% The dedication environment just starts a new page and prints the dedication
% in the center in italics.
%    \begin{macrocode}
\newenvironment{dedication}{
    \newpage
    \leavevmode\vfill
    \begin{center}
    \itshape
    }{
    \end{center}
    \vskip 3ex
    \vfill
    \newpage
    }
%    \end{macrocode}
% \end{environment}
%
% \begin{environment}{symbollist}
% \begin{environment}{symbollist*}
% The \env{symbollist} environments can be used to create a list of symbols or
% abbreviations. The starred version left-justifies the left column (good for
% lists of abbreviations) whereas the unstarred version centers the contents of
% the left column (good for lists of symbols).
%    \begin{macrocode}
\newenvironment*{symbollist}[1][1in]{
    \begin{list}{}{\singlespacing
     \setlength{\leftmargin}{#1}
     \setlength{\labelwidth}{#1}
     \addtolength{\labelwidth}{-\labelsep}
     \setlength{\topsep}{0in}}%
     \def\makelabel##1{\hfil##1\hfil}%
    }{
    \end{list}}
\newenvironment*{symbollist*}[1][1in]{
    \begin{symbollist}[#1]
    \def\makelabel##1{##1\hfil}}
    {\end{symbollist}}
%    \end{macrocode}
% \end{environment}
% \end{environment}
%
%
% \subsection{Chapter Headings}
%
% Text of chapter title must match exactly with text in Table of Contents.
% We support both plain chapter headings and ``fancy'' chapter headings.
%
% \begin{macro}{\chapternumberfont}
%    Define the font used for chapter numbers in fancy chapter headings.
%    If you're using scalable PostScript fonts, you might want to
%    override it, for example:
%    \begin{verbatim}
%    \renewcommand\chapternumberfont{
%       \fontseries{bx}\fontsize{72}{72}\selectfont}
%    \end{verbatim}
%    \begin{macrocode}
\if@thesisfancy
  \font\cminch=cminch at 60pt
  \newcommand\chapternumberfont{\cminch}
\else
  \newcommand\chapternumberfont{\huge\bfseries}
\fi
%    \end{macrocode}
% \end{macro}
%
% \begin{macro}{\chaptertitlefont}
%    Define the font used for chapter titles.
%    \begin{macrocode}
\newcommand\chaptertitlefont{\Huge\bfseries}
%    \end{macrocode}
% \end{macro}
%
% \begin{macro}{\@chapter}
%    This macro is called when we have a numbered chapter. When
%    |secnumdepth| is larger than $-1$ and, in the book
%    class, |\@mainmatter| is true, we display the chapter
%    number. We also inform the user that a new chapter is about to be
%    typeset by writing a message to the terminal. This definition
%    is the same as that in \pkg{book.cls} except that it makes different
%    entries in the table of contents for fancy chapter heads.
%    \begin{macrocode}
\def\@chapter[#1]#2{%
  \ifnum \c@secnumdepth >\m@ne
    \if@mainmatter
      \refstepcounter{chapter}%
      \typeout{\@chapapp\space\thechapter.}%
      \if@thesisfancy
        \addcontentsline{toc}{chapter}%
          {\protect\numberline{\thechapter}#1}%
      \else
        \addcontentsline{toc}{chapter}%
          {\@chapapp\ \thechapter\quad #1}%
      \fi
    \else
      \addcontentsline{toc}{chapter}{#1}%
    \fi
  \else
    \addcontentsline{toc}{chapter}{#1}%
  \fi
%    \end{macrocode}
%    After having written an entry to the table of contents we store
%    the (alternative) title of this chapter with |\chaptermark| and
%    add some white space to the lists of figures and tables.
%    \begin{macrocode}
  \chaptermark{#1}%
  \addtocontents{lof}{\protect\addvspace{10\p@}}%
  \addtocontents{lot}{\protect\addvspace{10\p@}}%
%    \end{macrocode}
%    Then we call upon |\@makechapterhead| to format the actual
%    chapter title. We have to do this in a special way when we are in
%    twocolumn mode in order to have the chapter title use the entire
%    |\textwidth|. In one column mode we call |\@afterheading|, which
%    takes care of suppressing the indentation.
%    \begin{macrocode}
  \if@twocolumn
    \@topnewpage[\@makechapterhead{#2}]%
  \else
    \@makechapterhead{#2}%
    \@afterheading
  \fi}
%    \end{macrocode}
% \end{macro}
%
% For fancy chapter headings, compute the correct height to use for the
% chapter number.  I want the chapter number to be centered on the first line
% of the chapter title.  If $a$ is the height of the chapter number and $b$ is
% the height of the chapter title, then if we set the chapter number in a
% box of height $b+(a-b)/2=(a+b)/2$ then it aligns correctly.
%
% We arrange for this value to be computed at the beginning of the document
% in case the user loads a style file that changed the default fonts.
%
% In addition, we want the chapter titles to have the same vertical placement
% on the page, regardless whether the chapter is numbered or not. We compute
% the distance we have to skip for chapters without numbers to accomplish this
% and store it in |\thesis@chapskip|.
%    \begin{macrocode}
\newskip\thesis@chapskip
\AtBeginDocument{%
  \newdimen\chapternumberheight
  \begingroup
    \chapternumberfont
    \setbox255=\hbox{A}
    \if@thesisfancy
      \global\thesis@chapskip=\ht255
    \else
      \global\thesis@chapskip=\baselineskip
    \fi
    \dimen255=\ht255
    \chaptertitlefont
    \setbox255=\hbox{A}
    \advance\dimen255 by \ht255
    \if@thesisfancy
      \global\advance\thesis@chapskip by -\ht255
      \global\divide\thesis@chapskip by 2
      \global\advance\thesis@chapskip by 10\p@
    \else
      \global\advance\thesis@chapskip by 20\p@
    \fi
    \divide\dimen255 by 2
    \global\chapternumberheight=\dimen255
  \endgroup}
%    \end{macrocode}
%
% \begin{macro}{\chaptertitleheight}
% The amount of space allotted for the chapter titles is stored in |\chaptertitleheight|.
% In this manner, the chapter text always appears at the same vertical place for each
% chapter, even if the title spills over into multiple lines.
%    \begin{macrocode}
\newlength{\chaptertitleheight}
\if@thesisfancy
  \setlength{\chaptertitleheight}{1.5in}
\else
  \setlength{\chaptertitleheight}{1.85in}
\fi
%    \end{macrocode}
% \end{macro}
%
% \begin{macro}{\@makechapterhead}
%    The macro |\@chapter| uses |\@makechapterhead|\meta{text} to format the
%    heading of the chapter.  This is a modified version of the standard
%    |\@makechapterhead|.  It sets the chapter heading in single spacing,
%    and it handles the fancy heading style. The whole heading is placed
%    in a |\vbox| so that it is confined to the spacing allotted to it
%    as defined in |\chaptertitleheight|.
%    \begin{macrocode}
\def\@makechapterhead#1{%
  \vbox to \chaptertitleheight{
    \def\baselinestretch{1}\@normalsize
    \parindent \z@ \raggedright \normalfont
    \if@centerchapter
      \centering
    \fi
    \ifnum \c@secnumdepth >\m@ne
      \if@mainmatter
        \thesis@chapskip=\z@
        \if@thesisfancy
          \vspace*{10\p@}%
          \leavevmode\llap{\vbox to \chapternumberheight{\hbox{%
            \chapternumberfont\thechapter\,}\vss}}%
        \else
          {\chapternumberfont \@chapapp\space \thechapter}
          \par\nobreak
          \vskip 20\p@
        \fi
      \fi
    \fi
    \interlinepenalty\@M
    \vspace*{\thesis@chapskip}%
    \chaptertitlefont #1
    \vfil
  }%
  \par\nobreak%
  }
%    \end{macrocode}
% \end{macro}
%
% \begin{macro}{\@makeschapterhead}
%    The macro |\@schapter| uses |\@makeschapterhead|\meta{text}to format
%    the heading of the chapter. It is similar to |\@makechapterhead|
%    except that it never has to print a chapter number.
%
%    \begin{macrocode}
\def\@makeschapterhead#1{%
  \vbox to \chaptertitleheight{
    \def\baselinestretch{1}\@normalsize
    \parindent \z@ \raggedright \normalfont
    \if@centerchapter
      \centering
    \fi
    \interlinepenalty\@M
    \vspace*{\thesis@chapskip}
    \chaptertitlefont #1
    \vfil
  }%
  \par\nobreak%
  }
%    \end{macrocode}
% \end{macro}
%
% \subsection{Lower Level Headings}
%
% \begin{macro}{\sectiontitlefont}
% \begin{macro}{\subsectiontitlefont}
% \begin{macro}{\subsubsectiontitlefont}
% These macros contain the font declarations for the sectioning titles.
%    \begin{macrocode}
\newcommand{\sectiontitlefont}{\Large\bfseries}
\newcommand{\subsectiontitlefont}{\large\bfseries}
\newcommand{\subsubsectiontitlefont}{\normalsize\bfseries}
%    \end{macrocode}
% \end{macro}
% \end{macro}
% \end{macro}
%
% \begin{macro}{\section}
% \begin{macro}{\subsection}
% \begin{macro}{\subsubsection}
% We redefine the lower level headings to set their titles
% ragged right.
% We don't have to change sectioning commands below
% \env{subsubsection} because they produce run-in headings.
%    \begin{macrocode}
\renewcommand\section{\@startsection {section}{1}{\z@}%
  {-3.5ex \@plus -1ex \@minus -.2ex}%
  {2.3ex \@plus.2ex}%
  {\raggedright\normalfont\sectiontitlefont}}
\renewcommand\subsection{\@startsection{subsection}{2}{\z@}%
  {-3.25ex\@plus -1ex \@minus -.2ex}%
  {1.5ex \@plus .2ex}%
  {\raggedright\normalfont\subsectiontitlefont}}
\renewcommand\subsubsection{\@startsection{subsubsection}{3}{\z@}%
  {-3.25ex\@plus -1ex \@minus -.2ex}%
  {1.5ex \@plus .2ex}%
  {\raggedright\normalfont\subsubsectiontitlefont}}
%    \end{macrocode}
% \end{macro}
% \end{macro}
% \end{macro}
%
% \subsection{Appendices}
%
% \begin{macro}{\appendix}
% Redefine the |\appendix| macro so that it can take a star
% if unlettered appendices are desired.
%    \begin{macrocode}
\let\thesis@appendix\appendix
\renewcommand\appendix{\thesis@appendix\@ifstar{\gdef\thechapter{}}{}}
%    \end{macrocode}
% \end{macro}
%
% \subsection{Bibliography}
%
% \begin{macro}{\bibname}
% UIUC Thesis format says that if references are cited as ``[1]'' then
% one of the terms ``References,'' ``List of References,'' or
% ``Literature Cited'' should be used instead of ``Bibliography.''
%    \begin{macrocode}
\renewcommand\bibname{References}
%    \end{macrocode}
% \end{macro}
%
% \begin{environment}{thebibliography}
% The standard definition of \env{thebibliography} environment issues the |\chapter*|
% command, which does not make the necessary entry to the TOC. Here the environment
% is redefined so that the unstarred version is used instead. In addition,
% the environment is also single-spaced for aesthetics. These modifications are done
% at the beginning of the document since some packages (\pkg{natbib} in particular)
% change the definition of \env{thebibliography} environment.
%    \begin{macrocode}
\AtBeginDocument{\let\thesis@thebib\thebibliography
    \let\thesis@endbib\endthebibliography
    \def\thebibliography{\begingroup\singlespacing%
        \chapter{\bibname}%
        \let\chapter\@gobbletwo%
        \thesis@thebib}
    \def\endthebibliography{\thesis@endbib\endgroup}}
%    \end{macrocode}
% \end{environment}
%
%
% \subsection{Index}
%
% \begin{environment}{theindex}
% The index is single spaced and a line is added to the Table of Contents.
%    \begin{macrocode}
\let\thesis@theindex=\theindex
\def\theindex{\addcontentsline{toc}{chapter}{\indexname}%
    \begingroup\singlespacing\thesis@theindex}
\let\thesis@endtheindex=\endtheindex
\def\endtheindex{\thesis@endtheindex\endgroup}
%    \end{macrocode}
% \end{environment}
%
%
% \subsection{Page Layout}
%
% First we set the vertical layout.
% Adjust the height of the text column so that it takes up the full
% height of an 8.5 by 11 inch page.
%    \begin{macrocode}
\topmargin=0pt
\advance \topmargin by -\headheight
\advance \topmargin by -\headsep
\textheight 8.9in
%    \end{macrocode}
% Next, set the horizontal layout.
%
% The standard for technical papers seems to be to use
% extremely wide columns of text, and then to increase the spacing between
% lines to compensate for the long lines.
% Unfortunately, because so many papers are typeset this way,
% the format has become self-propagating.
%
% The |[fullpage]| option sets one-inch margins.
%    \begin{macrocode}
\if@fullpage
  \setlength{\textwidth}{\paperwidth}
  \addtolength{\textwidth}{-2in}
  \@settopoint\textwidth
\fi
%    \end{macrocode}
% In the old version of \pkg{uiucthesis2009}, (past versions used the name uiucthesis)
% the ``fancy'' thesis style used
% an asymmetric page layout,
% shifting the text column slightly over to the right to leave
% room for the chapter number to the left of the chapter title. This layout
% is still used if the |[offcenter]| option is given, otherwise symmetric
% margins are used.
%    \begin{macrocode}
\setlength{\@tempdima}{\paperwidth}
\addtolength{\@tempdima}{-\textwidth}
\setlength{\oddsidemargin}{.5\@tempdima}
\addtolength{\oddsidemargin}{-1in}
\if@thesisoffcenter
  \addtolength{\oddsidemargin}{0.5in}
  \addtolength{\textwidth}{-0.5in}
  \reversemarginpar
\fi
%    \end{macrocode}
% Set |\marginparwidth|, leaving 24pt for the right margin.
% Note that you're not allowed to actually use a marginal paragraph
% this close to the edge in the final version of a thesis, but it is still
% handy for leaving notes to yourself in the draft (with the |\note|
% command, see below).
%    \begin{macrocode}
\setlength{\marginparwidth}{\oddsidemargin}
\addtolength{\marginparwidth}{1in}
\addtolength{\marginparwidth}{-\marginparsep}
\addtolength{\marginparwidth}{-24pt}
%    \end{macrocode}
% Use the same margins for even and odd pages. Use the \LaTeX macro
% |\@settopoint| to truncate arithmetic errors.
%    \begin{macrocode}
\@settopoint\oddsidemargin
\@settopoint\marginparwidth
\setlength{\evensidemargin}{\oddsidemargin}
%    \end{macrocode}
%
%
% \subsection{Making Notes}
%
% \begin{macro}{\note}
% You can leave yourself marginal notes using the |\note{|\meta{text}|}|
% macro. If the final draft is being printed (i.e., no |[draftthesis]| option)
% then these notes are not printed.
%    \begin{macrocode}
\if@thesisdraft
    \newcommand{\note}[1]{\marginpar{\def\baselinestretch{1}\small\raggedright #1}}
\else
    \newcommand{\note}[1]{}
    \let\thesis@marginpar\marginpar
    \def\marginpar{%
%<class>        \ClassWarning{uiucthesis2009}{Margin paragraphs fall outside the allowed margins\MessageBreak
%<package>        \PackageWarning{uiucthesis2009}{Margin paragraphs fall outside the allowed margins\MessageBreak
            for UIUC Theses, use \protect\note\space instead of \protect\marginpar.}%
        \thesis@marginpar}
\fi
%    \end{macrocode}
% \end{macro}
%
% \subsection{Page Numbering}
%
% Page numbers must be in one of three places, and must appear in the
% same place on \emph{every page}, including chapter openings.
%
% To accommodate the draft heading, we redefine the plain page style.
% \begin{macro}{\ps@plain}
%    \begin{macrocode}
\def\ps@plain{%
  \let\@mkboth\@gobbletwo
  \if@thesisdraft
    \def\@oddhead{\draftheader\hfil}
  \else
    \let\@oddhead\@empty
  \fi
  \let\@evenhead\@oddhead
  \def\@oddfoot{\reset@font\hfil\thepage\hfil}%
  \let\@evenfoot\@oddfoot
}
%    \end{macrocode}
% \end{macro}
%
% \begin{macro}{\ps@headings}
% The ``headings'' page style is also supported. The heading at the top
% will be within the 1 inch margin that you are supposed to allow, however.
% If the |[draftthesis]| option is given, there is probably not enough room
% for both the chapter number and title, so just print the number in that case.
%    \begin{macrocode}
\if@twoside
  \def\ps@headings{%
    \if@thesisdraft
      \def\@oddhead{\draftheader\hfil\slshape\rightmark}
      \def\@evenhead{\slshape\leftmark\hfil\draftheader}
    \else
      \def\@oddhead{\hfil\slshape\rightmark}
      \def\@evenhead{\slshape\leftmark\hfil}
    \fi
    \def\@oddfoot{\reset@font\hfil\thepage\hfil}%
    \let\@evenfoot\@oddfoot
    \let\@mkboth\markboth
    \if@thesisdraft
      \def\chaptermark##1{%
        \markboth {\MakeUppercase{%
          \ifnum \c@secnumdepth >\m@ne
            \if@mainmatter
              \@chapapp\ \thechapter%
            \fi
          \fi}}{}}
    \else
      \def\chaptermark##1{%
        \def\@chaphead{\MakeUppercase{%
          \ifnum \c@secnumdepth >\m@ne
            \if@mainmatter
              \if@thesisfancy
                \thechapter.~~%
              \else
                \@chapapp\ \thechapter.~~%
              \fi
            \fi
          \fi
          ##1}}%
        \markboth{\@chaphead}{\@chaphead}}
    \fi
    \def\sectionmark##1{%
      \markright {\MakeUppercase{%
        \ifnum \c@secnumdepth >\z@
          \thesection. \ %
        \fi
        ##1}}}}
\else
  \def\ps@headings{%
    \if@thesisdraft
      \def\@oddhead{\draftheader\hfil\slshape\rightmark}
    \else
      \def\@oddhead{\hfil\slshape\rightmark\hfil}
    \fi
    \let\@evenhead\@oddhead
    \def\@oddfoot{\reset@font\hfil\thepage\hfil}%
    \let\@evenfoot\@oddfoot
    \let\@mkboth\markboth
    \if@thesisdraft
      \def\chaptermark##1{%
        \markright {\MakeUppercase{%
          \ifnum \c@secnumdepth >\m@ne
            \if@mainmatter
              \@chapapp\ \thechapter%
            \fi
          \fi}}}
    \else
      \def\chaptermark##1{%
        \markright {\MakeUppercase{%
          \ifnum \c@secnumdepth >\m@ne
            \if@mainmatter
              \if@thesisfancy
                \thechapter.~~%
              \else
                \@chapapp\ \thechapter.~~%
              \fi
            \fi
          \fi
          ##1}}}
    \fi
     }
\fi
%    \end{macrocode}
% \end{macro}
%
% Set the default page style to (our new definition of) plain.
%    \begin{macrocode}
\pagestyle{plain}
%    \end{macrocode}
%
% \begin{macro}{\chapter}
% Redefine |\chapter| to not do |\thispagestyle{empty}| because
% even chapter openings should have page numbers in UIUC theses.
%    \begin{macrocode}
\renewcommand\chapter{\if@openright\cleardoublepage\else\clearpage\fi
  \@mkboth{}{}
  \thispagestyle{plain}
  \global\@topnum\z@
  \@afterindentfalse
  \secdef\@chapter\@schapter}
%    \end{macrocode}
% \end{macro}
%
%
% \subsection{Vita}
%
% The support for |\vita| is pretty minimal.
%
% \begin{macro}{\vitaname}
% Define |\vitaname| so the user can change it if he wants.
%    \begin{macrocode}
\newcommand\vitaname{Vita}
%    \end{macrocode}
% \end{macro}
%
% \begin{macro}{\vita}
% Vita should appear in Table of Contents, but should not be numbered.
%    \begin{macrocode}
\newcommand\vita{
  \chapter{\vitaname}%
}
%    \end{macrocode}
% \end{macro}
%
% \subsection{Body Formatting}
%
% \begin{macro}{\thesisspacing}
% The |\thesisspacing| command is called to switch to the default
% line spacing for the thesis.  The thesis format requirements
% require at least line and a half spacing.
% The \pkg{uiucthesis2009} class by default uses |\onehalfspacing|,
% or |\doublespacing| if the |[fullpage]| option is in effect.
%    \begin{macrocode}
\def\thesisspacing{\if@fullpage\doublespacing\else\onehalfspacing\fi}
%    \end{macrocode}
% \end{macro}
%
% At this point, we're ready to set up the actual formatting for the
% front matter of the thesis.  We use roman page numbers.
% Also, arrange so that |\thesisspacing| gets called when the document
% begins.  We don't just call it here because that wouldn't give the
% user a chance to override it.
%    \begin{macrocode}
\pagenumbering{roman}
\AtBeginDocument{\thesisspacing}
%    \end{macrocode}
%
% \subsection{Compatibility}
%
% \begin{macro}{preliminary}
% The old \pkg{uiucthesis2009} style defined a \env{preliminary}
% environment for the front matter.  This isn't needed with this
% style, so we redefine it to call |\frontmatter| for compatibility's sake.
%    \begin{macrocode}
\def\preliminary{\frontmatter}
\let\endpreliminary=\relax
%    \end{macrocode}
% \end{macro}
%
% \begin{macro}{thesis}
% Similarly, the old \pkg{uiucthesis2009} style defined a \env{thesis}
% environment that has been superceded by the |\mainmatter| command.
% We define it here for backward compatibility.
%    \begin{macrocode}
\def\thesis{\mainmatter}
\let\endthesis=\relax
%    \end{macrocode}
% \end{macro}
%
% \iffalse
%</class|package>
% \fi
% \Finale
\endinput
